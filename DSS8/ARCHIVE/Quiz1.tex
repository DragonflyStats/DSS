\documentclass[DSS8main.tex]{subfiles}
\begin{document}
\section{Week 1}
%-------------------------------------------------------------------------------------%
\subsection*{Question 1}
Which of the following are steps in building a machine learning algorithm?

\begin{itemize}
\item[i] Artificial intelligence
\item[ii] Data mining
\item[iii] Training and test sets
\item[iv] Creating features.
\end{itemize}
%-------------------------------------------------------------------------------------%
\subsection*{Question 2}
Suppose we build a prediction algorithm on a data set and it is 100\% accurate on that data set. Why might the algorithm not work well if we collect a new data set?
\begin{itemize}
\item[i] We may be using bad variables that don't explain the outcome.v
\item[ii] We have too few predictors to get good out of sample accuracy.
\item[iii] Our algorithm may be overfitting the training data, predicting both the signal and the noise.
\item[iv] We may be using a bad algorithm that doesn't predict well on this kind of data.
\end{itemize}
%-------------------------------------------------------------------------------------%
\subsection*{Question 3}
What are typical sizes for the training and test sets?
\begin{itemize}
\item[i] 90\% training set, 10\% test set
\item[ii] 60\% in the training set, 40\% in the testing set.
\item[iii] 20\% test set, 80\% training set.
\item[iv] 0\% training set, 100\% test set.
\end{itemize}
%-------------------------------------------------------------------------------------%
\subsection*{Question 4}
What are some common error rates for predicting binary variables (i.e. variables with two possible values like yes/no, disease/normal, clicked/didn't click)?
\begin{itemize}
\item[i] $R^2$
\item[ii] Median absolute deviation
\item[iii] Sensitivity
\item[iv] P-values
\end{itemize}
%-------------------------------------------------------------------------------------%
\subsection*{Question 5}
Suppose that we have created a machine learning algorithm that predicts whether a link will be clicked with 99\% sensitivity and 99\% specificity. The rate the link is clicked is 1/1000 of visits to a website. If we predict the link will be clicked on a specific visit, what is the probability it will actually be clicked?

\begin{itemize}
\item[i] 89.9\%
\item[ii] 0.009\%
\item[iii] 50\%
\item[iv] 9\%
\end{itemize}
%-------------------------------------------------------------------------------------%

\end{document}
