\documentclass[12pt]{article}


\usepackage{framed}

% http://rstudio-pubs-static.s3.amazonaws.com/1940_f7997f0ead914ae897a445f5c94bc96f.html

%https://github.com/pcward/coursera_programming-r/blob/master/rankhospital.r

\begin{document}


\section{Quiz for Week 3}

\subsection*{Using the help files}

\begin{itemize}
\item For many questions on this weeks quiz, reading the help files can help the user
determine the correct answers quickly.

\item Some of the options are not the actual names of the R commands.

\item It is also useful to learn about other commands, even if they are obviously not the correct answer
\end{itemize}
\begin{framed}
\begin{verbatim}
help(plot)
help(hist)
\end{verbatim}
\end{framed}
%------------------------------------------------------%
\subsection*{The lattice package}

%------------------------------------------------------%
\newpage
\subsection*{Question 1}
Which of these functions opens a graphics device in R?
\begin{itemize}
\item[(a)] save()
\item[(b)] pdf()
\item[(c)] axis()
\item[(d)] serialize()
\end{itemize}
%------------------------------------------------------%
\subsection*{Question 2}
Which function opens the default graphics device on Windows?
\begin{itemize}
\item[(a)] xfig()
\item[(b)] jpeg()
\item[(c)] windows()
\item[(d)] postscript()
\end{itemize}
%------------------------------------------------------%
\subsection*{Question 3}
Which of the following functions is part of the base graphics system?

\begin{itemize}
\item[(a)] barchart()
\item[(b)] histogram()
\item[(c)] xyplot()
\item[(d)] plot()
\end{itemize}
%------------------------------------------------------%
\subsection*{Question 4}
Which of the following functions is generally used to annotate a plot in the base graphics system?
\begin{itemize}
\item[(a)] barplot()
\item[(b)] points()
\item[(c)] plot()
\item[(d)] hist()
\end{itemize}




%------------------------------------------------------%
\subsection*{Question 5}
What does the 'pch' option to \texttt{par()} control?


\begin{itemize}
\item[(a)] the plotting symbol/character in the lattice graphics system
\item[(b)] the line width in the base graphics system
\item[(c)] the plotting symbol/character in the base graphics system
\item[(d)] the orientation of the axis labels on the plot
\end{itemize}





%------------------------------------------------------%
\subsection*{Question 6}
Under the lattice graphics system, what do the primary plotting functions return?

\begin{itemize}
\item[(a)] an object of class 'lattice'
\item[(b)] an object of class 'plot'
\item[(c)] nothing; only a plot is made
\item[(d)] an object of class 'trellis'
\end{itemize}




%------------------------------------------------------%
\newpage
\subsection*{Question 7}
What is produced by the following \texttt{R} code?
\begin{framed}
\begin{verbatim}
library(nlme)
library(lattice)
xyplot(weight ~ Time | Diet, BodyWeight)
\end{verbatim}
\end{framed}

\begin{itemize}
\item[(a)] A set of 3 panels showing the relationship between weight and time for each rat.
\item[(b)] A set of 3 panels showing the relationship between weight and time for each diet.
\item[(c)] A set of 16 panels showing the relationship between weight and time for each rat.
\item[(d)] A set of 11 panels showing the relationship between weight and diet for each time.
\end{itemize}



\begin{verbatim}
> summary(BodyWeight)
     weight           Time            Rat      Diet  
 Min.   :225.0   Min.   : 1.00   2      : 11   1:88  
 1st Qu.:267.0   1st Qu.:15.00   3      : 11   2:44  
 Median :344.5   Median :36.00   4      : 11   3:44  
 Mean   :384.5   Mean   :33.55   1      : 11         
 3rd Qu.:511.2   3rd Qu.:50.00   8      : 11         
 Max.   :628.0   Max.   :64.00   5      : 11         
                                 (Other):110      

> levels(BodyWeight$Diet)
[1] "1" "2" "3"
>
> levels(BodyWeight$Rat)
 [1] "2"  "3"  "4"  "1"  "8"  "5"  "6"  "7" 
 [9] "11" "9"  "10" "12" "13" "15" "14" "16"

\end{verbatim}


%------------------------------------------------------%
\subsection*{Question 8}
Which of the following functions can be used to annotate a panel in a multi-panel lattice plot?

\begin{itemize}
\item[(a)] lines()
\item[(b)] axis()
\item[(c)] lpoints()
\item[(d)] text()
\end{itemize}
%------------------------------------------------------%
\subsection*{Question 9}
Which R code makes a plot with the Greek letter 'theta' in the title?

\begin{itemize}
\item[(a)] plot(0, 0, main = expression(theta))
\item[(b)] plot(0, 0, main = "theta")
\item[(c)] plot(0, 0, main = expression("theta")
\item[(d)] plot(0, 0, main = substitute(theta))
\end{itemize}
%------------------------------------------------------%
\newpage

\subsection*{The set.seed() command}

\subsection*{Question 10}
What is produced at the end of this snippet of \texttt{R} code?

\begin{framed}
\begin{verbatim}
set.seed(1)
rpois(5, 2)
\end{verbatim}
\end{framed}


\begin{itemize}
\item[(a)] A vector with the numbers 3.3, 2.5, 0.5, 1.1, 1.7
\item[(b)] A vector with the numbers 1, 4, 1, 1, 5
\item[(c)] It is impossible to tell because the result is random
\item[(d)] A vector with the numbers 1, 1, 2, 4, 1
\end{itemize}

The\texttt{ rpois()} function is related to generating random numbers. Therefore option (c) would seem like a plausible answer.
However the command is preceded by the \texttt{set.seed(1)} command.
%------------------------------------------------------%
\end{document}