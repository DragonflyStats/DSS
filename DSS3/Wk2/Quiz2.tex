\documentclass[12pt]{article}
\usepackage{framed}
\usepackage{graphicx}
\begin{document}
%---------------------------------------------------------------------------------------------%
\newpage
\subsection*{Question 1}
\textbf{Question}
\begin{itemize}
	\item Register an application with the Github API here 
	\begin{verbatim}
	https://github.com/settings/applications. 
	\end{verbatim}
	\item Access the API to get information on your instructors repositories 
	\item (hint: this is the url you want "https://api.github.com/users/jtleek/repos"). \item Use this data to find the time that the datasharing repo was created. \item What time was it created?
	\item This tutorial may be useful
	\begin{verbatim} (https://github.com/hadley/httr/blob/master/demo/oauth2-github.r).
	\end{verbatim}
	\item You may also need to run the code in the base R package and not R studio.
\end{itemize}

\noindent \textbf{Options} \\
\begin{itemize}
\item[(i)] 2012-06-20T18:39:06Z
\item[(ii)] 2014-03-05T16:11:46Z
\item[(iii)] 2014-01-04T21:06:44Z
\item[(iv)] 2013-11-07T13:25:07Z
\end{itemize}

\newpage
\subsection*{The sqldf package}
%---------------------------------------------------------------------------------------------%
\newpage
\subsection*{Question 2}
The \textbf{sqldf} package allows for execution of SQL commands on \texttt{R} data frames. We will use the \textbf{sqldf} package to practice the queries we might send with the \texttt{dbSendQuery} command in RMySQL. 

\bigskip
\noindent Download the American Community Survey data and load it into an R object called
\texttt{ acs}.

\begin{verbatim}
https://d396qusza40orc.cloudfront.net/getdata%2Fdata%2Fss06pid.csv 
\end{verbatim}

\noindent Which of the following commands will select only the data for the probability weights \texttt{pwgtp1} with ages less than 50?
\begin{itemize}
\item[(i)] \texttt{sqldf("select * from acs where AGEP < 50 and pwgtp1")}
\item[(ii)] \texttt{sqldf("select * from acs")}
\item[(iii)] \texttt{sqldf("select * from acs where AGEP < 50")}
\item[(iv)] \texttt{sqldf("select pwgtp1 from acs where AGEP < 50")}
\end{itemize}

\begin{framed}
\begin{verbatim}
names(acs)
\end{verbatim}
\end{framed}
%---------------------------------------------------------------------------------------------%
\newpage
\subsection*{Question 3}
Using the same data frame you created in the previous problem, what is the equivalent function to \texttt{unique(acs\$AGEP)}
\begin{itemize}
\item[(i)] \texttt{sqldf("select unique AGEP from acs")}
\item[(ii)] \texttt{sqldf("select distinct pwgtp1 from acs")}
\item[(iii)] \texttt{sqldf("select AGEP where unique from acs")}
\item[(iv)] \texttt{sqldf("select distinct AGEP from acs")}
\end{itemize}

%---------------------------------------------------------------------------------------------%
\newpage
\subsection*{Question 4}
How many characters are in the 10th, 20th, 30th and 100th lines of HTML from this page: 
\begin{verbatim}
http://biostat.jhsph.edu/~jleek/contact.html 
\end{verbatim}
(Hint: the \texttt{nchar()} function in \texttt{R} may be helpful)
\begin{itemize}
\item[(i)] 43 99 8 6
\item[(ii)] 45 31 7 31
\item[(iii)] 43 99 7 25
\item[(iv)] 45 31 7 25
\item[(v)] 45 0 2 2
\item[(vi)] 45 31 2 25
\item[(vii)] 45 92 7 2
\end{itemize}

%---------------------------------------------------------------------------------------------%
\newpage
\subsection*{Question 5}
Read this data set into R and report the sum of the numbers in the fourth column. 
\begin{verbatim}
https://d396qusza40orc.cloudfront.net/getdata%2Fwksst8110.for 
\end{verbatim} 

\noindent Original source of the data:
\begin{verbatim}
http://www.cpc.ncep.noaa.gov/data/indices/wksst8110.for 
\end{verbatim} 

\textit{(Hint this is a fixed width file (fwf) format)}
\begin{itemize}
\item 32426.7
\item 35824.9
\item 222243.1
\item 36.5
\item 28893.3
\item 101.83
\end{itemize}

\begin{framed}
\begin{verbatim}
help(read.fwf)
\end{verbatim}	
\end{framed}
\end{document}