% Options for packages loaded elsewhere
\PassOptionsToPackage{unicode}{hyperref}
\PassOptionsToPackage{hyphens}{url}
%
\documentclass[
]{article}
\usepackage{lmodern}
\usepackage{amsmath}
\usepackage{ifxetex,ifluatex}
\ifnum 0\ifxetex 1\fi\ifluatex 1\fi=0 % if pdftex
  \usepackage[T1]{fontenc}
  \usepackage[utf8]{inputenc}
  \usepackage{textcomp} % provide euro and other symbols
  \usepackage{amssymb}
\else % if luatex or xetex
  \usepackage{unicode-math}
  \defaultfontfeatures{Scale=MatchLowercase}
  \defaultfontfeatures[\rmfamily]{Ligatures=TeX,Scale=1}
\fi
% Use upquote if available, for straight quotes in verbatim environments
\IfFileExists{upquote.sty}{\usepackage{upquote}}{}
\IfFileExists{microtype.sty}{% use microtype if available
  \usepackage[]{microtype}
  \UseMicrotypeSet[protrusion]{basicmath} % disable protrusion for tt fonts
}{}
\makeatletter
\@ifundefined{KOMAClassName}{% if non-KOMA class
  \IfFileExists{parskip.sty}{%
    \usepackage{parskip}
  }{% else
    \setlength{\parindent}{0pt}
    \setlength{\parskip}{6pt plus 2pt minus 1pt}}
}{% if KOMA class
  \KOMAoptions{parskip=half}}
\makeatother
\usepackage{xcolor}
\IfFileExists{xurl.sty}{\usepackage{xurl}}{} % add URL line breaks if available
\IfFileExists{bookmark.sty}{\usepackage{bookmark}}{\usepackage{hyperref}}
\hypersetup{
  pdftitle={Regression: Quiz 1},
  hidelinks,
  pdfcreator={LaTeX via pandoc}}
\urlstyle{same} % disable monospaced font for URLs
\usepackage[margin=1in]{geometry}
\usepackage{color}
\usepackage{fancyvrb}
\newcommand{\VerbBar}{|}
\newcommand{\VERB}{\Verb[commandchars=\\\{\}]}
\DefineVerbatimEnvironment{Highlighting}{Verbatim}{commandchars=\\\{\}}
% Add ',fontsize=\small' for more characters per line
\usepackage{framed}
\definecolor{shadecolor}{RGB}{248,248,248}
\newenvironment{Shaded}{\begin{snugshade}}{\end{snugshade}}
\newcommand{\AlertTok}[1]{\textcolor[rgb]{0.94,0.16,0.16}{#1}}
\newcommand{\AnnotationTok}[1]{\textcolor[rgb]{0.56,0.35,0.01}{\textbf{\textit{#1}}}}
\newcommand{\AttributeTok}[1]{\textcolor[rgb]{0.77,0.63,0.00}{#1}}
\newcommand{\BaseNTok}[1]{\textcolor[rgb]{0.00,0.00,0.81}{#1}}
\newcommand{\BuiltInTok}[1]{#1}
\newcommand{\CharTok}[1]{\textcolor[rgb]{0.31,0.60,0.02}{#1}}
\newcommand{\CommentTok}[1]{\textcolor[rgb]{0.56,0.35,0.01}{\textit{#1}}}
\newcommand{\CommentVarTok}[1]{\textcolor[rgb]{0.56,0.35,0.01}{\textbf{\textit{#1}}}}
\newcommand{\ConstantTok}[1]{\textcolor[rgb]{0.00,0.00,0.00}{#1}}
\newcommand{\ControlFlowTok}[1]{\textcolor[rgb]{0.13,0.29,0.53}{\textbf{#1}}}
\newcommand{\DataTypeTok}[1]{\textcolor[rgb]{0.13,0.29,0.53}{#1}}
\newcommand{\DecValTok}[1]{\textcolor[rgb]{0.00,0.00,0.81}{#1}}
\newcommand{\DocumentationTok}[1]{\textcolor[rgb]{0.56,0.35,0.01}{\textbf{\textit{#1}}}}
\newcommand{\ErrorTok}[1]{\textcolor[rgb]{0.64,0.00,0.00}{\textbf{#1}}}
\newcommand{\ExtensionTok}[1]{#1}
\newcommand{\FloatTok}[1]{\textcolor[rgb]{0.00,0.00,0.81}{#1}}
\newcommand{\FunctionTok}[1]{\textcolor[rgb]{0.00,0.00,0.00}{#1}}
\newcommand{\ImportTok}[1]{#1}
\newcommand{\InformationTok}[1]{\textcolor[rgb]{0.56,0.35,0.01}{\textbf{\textit{#1}}}}
\newcommand{\KeywordTok}[1]{\textcolor[rgb]{0.13,0.29,0.53}{\textbf{#1}}}
\newcommand{\NormalTok}[1]{#1}
\newcommand{\OperatorTok}[1]{\textcolor[rgb]{0.81,0.36,0.00}{\textbf{#1}}}
\newcommand{\OtherTok}[1]{\textcolor[rgb]{0.56,0.35,0.01}{#1}}
\newcommand{\PreprocessorTok}[1]{\textcolor[rgb]{0.56,0.35,0.01}{\textit{#1}}}
\newcommand{\RegionMarkerTok}[1]{#1}
\newcommand{\SpecialCharTok}[1]{\textcolor[rgb]{0.00,0.00,0.00}{#1}}
\newcommand{\SpecialStringTok}[1]{\textcolor[rgb]{0.31,0.60,0.02}{#1}}
\newcommand{\StringTok}[1]{\textcolor[rgb]{0.31,0.60,0.02}{#1}}
\newcommand{\VariableTok}[1]{\textcolor[rgb]{0.00,0.00,0.00}{#1}}
\newcommand{\VerbatimStringTok}[1]{\textcolor[rgb]{0.31,0.60,0.02}{#1}}
\newcommand{\WarningTok}[1]{\textcolor[rgb]{0.56,0.35,0.01}{\textbf{\textit{#1}}}}
\usepackage{graphicx}
\makeatletter
\def\maxwidth{\ifdim\Gin@nat@width>\linewidth\linewidth\else\Gin@nat@width\fi}
\def\maxheight{\ifdim\Gin@nat@height>\textheight\textheight\else\Gin@nat@height\fi}
\makeatother
% Scale images if necessary, so that they will not overflow the page
% margins by default, and it is still possible to overwrite the defaults
% using explicit options in \includegraphics[width, height, ...]{}
\setkeys{Gin}{width=\maxwidth,height=\maxheight,keepaspectratio}
% Set default figure placement to htbp
\makeatletter
\def\fps@figure{htbp}
\makeatother
\setlength{\emergencystretch}{3em} % prevent overfull lines
\providecommand{\tightlist}{%
  \setlength{\itemsep}{0pt}\setlength{\parskip}{0pt}}
\setcounter{secnumdepth}{-\maxdimen} % remove section numbering
\ifluatex
  \usepackage{selnolig}  % disable illegal ligatures
\fi

\title{Regression: Quiz 1}
\author{}
\date{\vspace{-2.5em}}

\begin{document}
\maketitle

\hypertarget{quiz-1}{%
\subsection{Quiz 1}\label{quiz-1}}

This is Quiz 1 from Coursera's Regression Models class within the Data
Science Specialization. This publication is intended as a learning
resource, all answers are documented and explained. Datasets are
available in R packages.

\begin{center}\rule{0.5\linewidth}{0.5pt}\end{center}

\hypertarget{question-1.}{%
\paragraph{Question 1.}\label{question-1.}}

Consider the data set given below

\begin{Shaded}
\begin{Highlighting}[]
\NormalTok{x }\OtherTok{\textless{}{-}} \FunctionTok{c}\NormalTok{(}\FloatTok{0.18}\NormalTok{, }\SpecialCharTok{{-}}\FloatTok{1.54}\NormalTok{, }\FloatTok{0.42}\NormalTok{, }\FloatTok{0.95}\NormalTok{)}
\end{Highlighting}
\end{Shaded}

And weights given by

\begin{Shaded}
\begin{Highlighting}[]
\NormalTok{w }\OtherTok{\textless{}{-}} \FunctionTok{c}\NormalTok{(}\DecValTok{2}\NormalTok{, }\DecValTok{1}\NormalTok{, }\DecValTok{3}\NormalTok{, }\DecValTok{1}\NormalTok{)}
\end{Highlighting}
\end{Shaded}

Give the value of μ that minimizes the least squares equation

\[\sum^{n}_{i=1}w_i(x_i−\mu)^2\]

\hypertarget{explanation}{%
\paragraph{Explanation:}\label{explanation}}

Least squares is minimized by the mean

\begin{Shaded}
\begin{Highlighting}[]
\FunctionTok{mean}\NormalTok{(w}\SpecialCharTok{*}\NormalTok{x)}\SpecialCharTok{/}\FunctionTok{mean}\NormalTok{(w)}
\end{Highlighting}
\end{Shaded}

\begin{verbatim}
## [1] 0.1471429
\end{verbatim}

\begin{center}\rule{0.5\linewidth}{0.5pt}\end{center}

\hypertarget{question-2.}{%
\paragraph{Question 2.}\label{question-2.}}

Consider the following data set

\begin{Shaded}
\begin{Highlighting}[]
\NormalTok{x }\OtherTok{\textless{}{-}} \FunctionTok{c}\NormalTok{(}\FloatTok{0.8}\NormalTok{, }\FloatTok{0.47}\NormalTok{, }\FloatTok{0.51}\NormalTok{, }\FloatTok{0.73}\NormalTok{, }\FloatTok{0.36}\NormalTok{, }\FloatTok{0.58}\NormalTok{, }\FloatTok{0.57}\NormalTok{, }\FloatTok{0.85}\NormalTok{, }\FloatTok{0.44}\NormalTok{, }\FloatTok{0.42}\NormalTok{)}
\NormalTok{y }\OtherTok{\textless{}{-}} \FunctionTok{c}\NormalTok{(}\FloatTok{1.39}\NormalTok{, }\FloatTok{0.72}\NormalTok{, }\FloatTok{1.55}\NormalTok{, }\FloatTok{0.48}\NormalTok{, }\FloatTok{1.19}\NormalTok{, }\SpecialCharTok{{-}}\FloatTok{1.59}\NormalTok{, }\FloatTok{1.23}\NormalTok{, }\SpecialCharTok{{-}}\FloatTok{0.65}\NormalTok{, }\FloatTok{1.49}\NormalTok{, }\FloatTok{0.05}\NormalTok{)}
\end{Highlighting}
\end{Shaded}

Fit the regression through the origin and get the slope treating \(y\)
as the outcome and \(X\) as the regressor.

(\textbf{Hint, do not center the data since we want regression through
the origin, not through the means of the data.})

\hypertarget{explanation-1}{%
\paragraph{Explanation:}\label{explanation-1}}

Subtract 1 from the regressor in ln to fit line through origin.

\begin{Shaded}
\begin{Highlighting}[]
\FunctionTok{summary}\NormalTok{(}\FunctionTok{lm}\NormalTok{(y}\SpecialCharTok{\textasciitilde{}}\NormalTok{x}\DecValTok{{-}1}\NormalTok{))}
\end{Highlighting}
\end{Shaded}

\begin{verbatim}
## 
## Call:
## lm(formula = y ~ x - 1)
## 
## Residuals:
##     Min      1Q  Median      3Q     Max 
## -2.0692 -0.2536  0.5303  0.8592  1.1286 
## 
## Coefficients:
##   Estimate Std. Error t value Pr(>|t|)
## x   0.8263     0.5817   1.421    0.189
## 
## Residual standard error: 1.094 on 9 degrees of freedom
## Multiple R-squared:  0.1831, Adjusted R-squared:  0.09238 
## F-statistic: 2.018 on 1 and 9 DF,  p-value: 0.1892
\end{verbatim}

\begin{center}\rule{0.5\linewidth}{0.5pt}\end{center}

\hypertarget{question-3.}{%
\paragraph{Question 3.}\label{question-3.}}

Do 𝚍𝚊𝚝𝚊(𝚖𝚝𝚌𝚊𝚛𝚜) from the datasets package and fit the regression model
with mpg as the outcome and weight as the predictor. Give the slope
coefficient.

\hypertarget{explanation-2}{%
\paragraph{Explanation:}\label{explanation-2}}

Predicting with the lower and upper bounds of the confidence intervals

\begin{Shaded}
\begin{Highlighting}[]
\NormalTok{fit }\OtherTok{\textless{}{-}} \FunctionTok{lm}\NormalTok{(mpg}\SpecialCharTok{\textasciitilde{}}\NormalTok{wt,mtcars)}
\FunctionTok{summary}\NormalTok{(fit)}\SpecialCharTok{$}\NormalTok{coef}
\end{Highlighting}
\end{Shaded}

\begin{verbatim}
##              Estimate Std. Error   t value     Pr(>|t|)
## (Intercept) 37.285126   1.877627 19.857575 8.241799e-19
## wt          -5.344472   0.559101 -9.559044 1.293959e-10
\end{verbatim}

\begin{center}\rule{0.5\linewidth}{0.5pt}\end{center}

\hypertarget{question-4.}{%
\paragraph{Question 4.}\label{question-4.}}

Consider data with an outcome (Y) and a predictor (X). The standard
deviation of the predictor is one half that of the outcome. The
correlation between the two variables is 0.5. What value would the slope
coefficient for the regression model with Y as the outcome and \(X\) as
the predictor?

\hypertarget{explanation-3}{%
\paragraph{Explanation:}\label{explanation-3}}

\begin{Shaded}
\begin{Highlighting}[]
\CommentTok{\# Correlation(XY)* SDy/SDx}
\FloatTok{0.5}\SpecialCharTok{*}\NormalTok{(}\DecValTok{1}\SpecialCharTok{/}\FloatTok{0.5}\NormalTok{)}
\end{Highlighting}
\end{Shaded}

\begin{verbatim}
## [1] 1
\end{verbatim}

\begin{center}\rule{0.5\linewidth}{0.5pt}\end{center}

\hypertarget{question-5.}{%
\paragraph{Question 5.}\label{question-5.}}

Students were given two hard tests and scores were normalized to have
empirical mean 0 and variance 1. The correlation between the scores on
the two tests was 0.4. What would be the expected score on Quiz 2 for a
student who had a normalized score of 1.5 on Quiz 1?

\hypertarget{explanation-4}{%
\paragraph{Explanation:}\label{explanation-4}}

Since the distributions are normalized, the slope coefficient is equal
to the correlation.

\begin{Shaded}
\begin{Highlighting}[]
\FloatTok{1.5}\SpecialCharTok{*}\FloatTok{0.4}
\end{Highlighting}
\end{Shaded}

\begin{verbatim}
## [1] 0.6
\end{verbatim}

\begin{center}\rule{0.5\linewidth}{0.5pt}\end{center}

\hypertarget{question-6.}{%
\paragraph{Question 6.}\label{question-6.}}

Consider the data given by the following

\begin{Shaded}
\begin{Highlighting}[]
\NormalTok{x }\OtherTok{\textless{}{-}} \FunctionTok{c}\NormalTok{(}\FloatTok{8.58}\NormalTok{, }\FloatTok{10.46}\NormalTok{, }\FloatTok{9.01}\NormalTok{, }\FloatTok{9.64}\NormalTok{, }\FloatTok{8.86}\NormalTok{)}
\end{Highlighting}
\end{Shaded}

What is the value of the first measurement if \(X\) were normalized (to
have mean 0 and variance 1)?

\hypertarget{explanation-5}{%
\paragraph{Explanation:}\label{explanation-5}}

To normalize, we subtract the mean and divide by the standard deviation.

\begin{Shaded}
\begin{Highlighting}[]
\NormalTok{(x}\SpecialCharTok{{-}}\FunctionTok{mean}\NormalTok{(x))}\SpecialCharTok{/}\FunctionTok{sd}\NormalTok{(x)}
\end{Highlighting}
\end{Shaded}

\begin{verbatim}
## [1] -0.9718658  1.5310215 -0.3993969  0.4393366 -0.5990954
\end{verbatim}

\begin{Shaded}
\begin{Highlighting}[]
\FunctionTok{scale}\NormalTok{(x)}
\end{Highlighting}
\end{Shaded}

\begin{verbatim}
##            [,1]
## [1,] -0.9718658
## [2,]  1.5310215
## [3,] -0.3993969
## [4,]  0.4393366
## [5,] -0.5990954
## attr(,"scaled:center")
## [1] 9.31
## attr(,"scaled:scale")
## [1] 0.7511325
\end{verbatim}

\begin{center}\rule{0.5\linewidth}{0.5pt}\end{center}

\hypertarget{question-7.}{%
\paragraph{Question 7.}\label{question-7.}}

Consider the following data set (used above as well). What is the
intercept for fitting the model with \(X\) as the predictor and y as the
outcome?

\begin{Shaded}
\begin{Highlighting}[]
\NormalTok{x }\OtherTok{\textless{}{-}} \FunctionTok{c}\NormalTok{(}\FloatTok{0.8}\NormalTok{, }\FloatTok{0.47}\NormalTok{, }\FloatTok{0.51}\NormalTok{, }\FloatTok{0.73}\NormalTok{, }\FloatTok{0.36}\NormalTok{, }\FloatTok{0.58}\NormalTok{, }\FloatTok{0.57}\NormalTok{, }\FloatTok{0.85}\NormalTok{, }\FloatTok{0.44}\NormalTok{, }\FloatTok{0.42}\NormalTok{)}
\NormalTok{y }\OtherTok{\textless{}{-}} \FunctionTok{c}\NormalTok{(}\FloatTok{1.39}\NormalTok{, }\FloatTok{0.72}\NormalTok{, }\FloatTok{1.55}\NormalTok{, }\FloatTok{0.48}\NormalTok{, }\FloatTok{1.19}\NormalTok{, }\SpecialCharTok{{-}}\FloatTok{1.59}\NormalTok{, }\FloatTok{1.23}\NormalTok{, }\SpecialCharTok{{-}}\FloatTok{0.65}\NormalTok{, }\FloatTok{1.49}\NormalTok{, }\FloatTok{0.05}\NormalTok{)}
\end{Highlighting}
\end{Shaded}

\hypertarget{explanation-6}{%
\paragraph{Explanation:}\label{explanation-6}}

The slope coefficient represents the change in the outcome per unit
change in regressor. (outcome/regressor) So if you divide the regressor
(m -\textgreater{} cm) you are effectively multiplying the outcome by
shrinking the units. If you multiply the regressor it will have the
opposite effect. The actual change is not effected, only how it is
expressed relative to the units of the regressor.

\begin{Shaded}
\begin{Highlighting}[]
\FunctionTok{summary}\NormalTok{(}\FunctionTok{lm}\NormalTok{(y}\SpecialCharTok{\textasciitilde{}}\NormalTok{x))}
\end{Highlighting}
\end{Shaded}

\begin{verbatim}
## 
## Call:
## lm(formula = y ~ x)
## 
## Residuals:
##     Min      1Q  Median      3Q     Max 
## -2.1640 -0.5818  0.2010  0.6669  1.1928 
## 
## Coefficients:
##             Estimate Std. Error t value Pr(>|t|)
## (Intercept)    1.567      1.252   1.252    0.246
## x             -1.713      2.105  -0.814    0.439
## 
## Residual standard error: 1.061 on 8 degrees of freedom
## Multiple R-squared:  0.07642,    Adjusted R-squared:  -0.03903 
## F-statistic: 0.662 on 1 and 8 DF,  p-value: 0.4394
\end{verbatim}

\begin{Shaded}
\begin{Highlighting}[]
\FunctionTok{coef}\NormalTok{(}\FunctionTok{lm}\NormalTok{(y}\SpecialCharTok{\textasciitilde{}}\NormalTok{x))}
\end{Highlighting}
\end{Shaded}

\begin{verbatim}
## (Intercept)           x 
##    1.567461   -1.712846
\end{verbatim}

\begin{center}\rule{0.5\linewidth}{0.5pt}\end{center}

\hypertarget{question-8.}{%
\paragraph{Question 8.}\label{question-8.}}

You know that both the predictor and response have mean 0. What can be
said about the intercept when you fit a linear regression?

It must be identically 0.

\hypertarget{explanation-7}{%
\paragraph{Explanation:}\label{explanation-7}}

\begin{center}\rule{0.5\linewidth}{0.5pt}\end{center}

\hypertarget{question-9.}{%
\paragraph{Question 9.}\label{question-9.}}

Consider the data given by

\begin{Shaded}
\begin{Highlighting}[]
\NormalTok{x }\OtherTok{\textless{}{-}} \FunctionTok{c}\NormalTok{(}\FloatTok{0.8}\NormalTok{, }\FloatTok{0.47}\NormalTok{, }\FloatTok{0.51}\NormalTok{, }\FloatTok{0.73}\NormalTok{, }\FloatTok{0.36}\NormalTok{, }\FloatTok{0.58}\NormalTok{, }\FloatTok{0.57}\NormalTok{, }\FloatTok{0.85}\NormalTok{, }\FloatTok{0.44}\NormalTok{, }\FloatTok{0.42}\NormalTok{)}
\end{Highlighting}
\end{Shaded}

\hypertarget{explanation-8}{%
\paragraph{Explanation:}\label{explanation-8}}

For LS, the mean minimizes

\begin{Shaded}
\begin{Highlighting}[]
\FunctionTok{mean}\NormalTok{(x)}
\end{Highlighting}
\end{Shaded}

\begin{verbatim}
## [1] 0.573
\end{verbatim}

\begin{center}\rule{0.5\linewidth}{0.5pt}\end{center}

\hypertarget{question-10.}{%
\paragraph{Question 10.}\label{question-10.}}

Let the slope having fit Y as the outcome and \(X\) as the predictor be
denoted as \(\beta_1\). Let the slope from fitting \(X\) as the outcome
and Y as the predictor be denoted as γ1. Suppose that you divide
\(\beta_1\) by γ1; in other words consider \(\beta_1\)/γ1. What is this
ratio always equal to?

Var(Y)/Var(X)

\hypertarget{explanation-9}{%
\paragraph{Explanation:}\label{explanation-9}}

Beta = cor(X,Y) * SDy/SDx\ldots.

Since Cor(X,Y)=Cor(Y,X) they will cancel out leaving the standard
deviations squared (variance.)

\begin{center}\rule{0.5\linewidth}{0.5pt}\end{center}

\end{document}
